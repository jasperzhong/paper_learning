\documentclass{article}
    \usepackage{ctex}
    \title{alphago论文学习}
    \author{Yuchen Zhong}
    \begin{document}
        \maketitle

        \section{alphago}
            用传统的办法,主要是搜索,但是对于围棋这种depth和breadth都很大的棋类
            来说,是非常困难的。目前最强的传统方法只达到了业余水平。

            本文训练了4个神经网络:
            \begin{itemize}
                \item $p_{\sigma}$: 用Supervised learning学习人类棋谱
                \item $p_{\pi}$: 同上,但是神经网络更浅,速度更快,称为快速下子(rollouts)
                \item $p_{\rho}$: 强化学习,左右互搏来optimise
                \item $v_{\theta}$:价值网络,根据局面算出胜率,predict winner
            \end{itemize}

            第一步是监督学习,
            Supervised Learning主要是用了CNN训练一个策略网络,
            输入是19x19x48的tensor. 为什么维度
            这么高呢? 19x19是棋盘大小,48包括了各种信息(黑子,白子,打吃,征吃等等).
            网络结构是13层,最后一层是softmax,输出是各个可下点的概率值$p(a|s)$.

            然后用人类下的位置作为labels, 用的是随机梯度上升来训练.目标是maximize
            人类大师下的位置的概率值.权值$\sigma$用随机梯度上升的修改:

            \begin{equation}
                \Delta \sigma \propto \frac{\partial log p_{\sigma(a|s)}}{\partial \sigma}
            \end{equation}

            

            学习了3千万步的围棋走法后,这个策略网络可以模拟人类落子57\%正确率.但是
            缺点是速度太慢.所以还训练了一个更快的但是准确率所有降低的快速下子策略网络
            (rollout policy network).

            第二步是强化学习,是为了进一步提高policy network的能力,
            就是从和上一步最终得到的神经网络开始训练,然后随机从之前迭代中训练的
            某个策略网络进行对弈(随机是为了防止过拟合)
            
            虽然网络结构和SL相同,但是目标不再相同.SL的目标是模拟人类的下棋,但是
            RL的目的是为了赢棋. 它的方法是定义了一个reward function $r(s_t)$.
            T代表终止的步骤,当$t < T$时,那么$r(s_t)=0$ 每一步的收益定义为
            $z_t=\pm r(s_T)$
            如果赢棋,那么为+1;输棋就为-1.权值修改如下:

            \begin{equation}
                \Delta \rho \propto \frac{\partial log p_{\sigma(a_t|s_t)}}{\partial \rho}z_t
            \end{equation}

            乘以$z_t$是为了让网络朝着赢棋的方向训练.

            这样训练出来的RL策略网络,对战SL策略网络就有80\%的胜率了.而且对战传统
            搜索方法的围棋AI也有85\%的胜率.

            第三步是训练value network,
            首先当然是要定义什么是value,
            \begin{equation}
                v^{p}(s) = E[z_t|s_t=s,a_{t...T}\in p]
            \end{equation}
            意思就是给定策略方案p,从状态s出发一直到结束的期望收益

            当然我们是不知道最优的方案p是什么,所以只能用当前最强的RL策略
            网络来计算一个棋面的价值$ v^{p_\rho}(s)$.
            然后训练一个value network来拟合,即
            $v_\theta \approx v^{p_\rho}(s) \approx v^*(s)$

            Value network也是一个CNN,但是输出是一个神经元,也就是一个标量,
            代表价值.cost function就用MSE,然后用SGD来收敛.

            \begin{equation}
                \Delta \theta \propto = \frac{\partial v_\theta(s)}{\partial \theta}(z - v_\theta(s))
            \end{equation}

            第四步就是MCTS(蒙特卡洛搜索)了.
            MCTS主要分为4步:

            \begin{enumerate}
                \item selection:首先每个节点(论文中是边,这个没有区别)都记录三个值:
                \begin{itemize}
                    \item $Q(s,a)$action value
                    \item $N(s,a)$visit count
                    \item $P(s,a)$prior probability
                    计算公式后面会谈到.

                    每次模拟从当前局面开始.在第t步的时候,根据如下公式选择$a_t$
                    \begin{equation}
                        a_t = \arg\max_a(Q(s_t,a)+u(s_t,a))
                    \end{equation}

                    其中
                    \begin{equation}
                        u(s,a) \propto \frac{P(s,a)}{1+N(s,a)}
                    \end{equation}

                    $u(s,a)$用了一种反馈的思想,就是访问地越多,下轮中就越难访问,
                    而先验概率越大,下轮中就越容易访问.
                    (分母中的1是为了防止除零)
                \end{itemize}
                \item expansion:根据如上方法终止于叶子节点,局面为$s_L$.这时候要对叶子节点进行
                expansion.怎么拓展呢?使用SL策略网络算出$p(a|s_L)$,把这个顺便
                作为先验概率$P(s_L,a)$
                \item simulation:一方面用快速下子策略网络(rollout)一直模拟到棋局结束.
                得到一个反馈($z_L$),另一方面用value网络来评估出一个value,然后总的value
                是二者的加权平均(参数$\lambda$),
                \begin{equation}
                    V(s_L) = (1-\lambda)v_\theta(S_L) + \lambda z_L
                \end{equation}
                后文提到$\lambda$取值0.5表现最好
                \item backpropagation:最后把模拟路径上所有的节点的访问次数增加1,
                同时更新Q值,形式化描述如下:
                \begin{equation}
                    N(s,a) = \sum_{i=1}^{n}1(s,a,i) 
                \end{equation}
                \begin{equation}
                    Q(s,a) = \frac{1}{N(s,a)} \sum_{i=1}^{n}1(s,a,i)V(s_{L}^i)
                \end{equation}
                
                其中,$s_{L}^i$是第i次模拟中的叶子节点.
                
                模拟多次后,采用root下面访问最多的节点为下一步的走法.
                (理由很简单,访问次数和访问概率是成反比的,所以访问次数最多证明是best move)

                看到这里不免有个疑问,为什么不采用RL却用SL呢?文中提到,是因为SL performed better
                可能是因为人类是选择了几步有希望的棋,但是RL只选择当前最优.

                所以纵观alphago,主要是以MCTS为主要框架,SL来算先验概率,rollout和value network
                来评估局面, RL主要是用来产生训练value network的数据.

                \section{alphago zero}

                如果说alphago的算法比较累赘的话,那么alphago zero的算法极为简洁.
                总的来讲就是ResNet + MCTS 
                
                ResNet的权重参数为$\theta$,输入是19x19x17(比alphago降低了很多)的棋面,
                输出有两种值,一种为可能落子点的概率$p(a|s)$,另一个就是胜率$v(s)$.
                (看到这里,感觉就是把alphago的SL和value network 合并了)
                这个网络称为$f_\theta$

                这里要注意的是,既然不用人类棋谱,那么棋谱从哪来呢?答案就是机器自我对弈生成
                棋谱.当然一开始生成的棋谱都是很弱智的,不过在短短40天内,就能超越人类水平,
                进步还是非常神速.

                然后利用MCTS进行下棋,过程和alphago类似,不过在simulation那步,
                用$f_\theta$代替SL.同样,输出的概率作为先验概率,但是评估就直接用
                $f_\theta$算出来的v了.后面过程和alphago一样.

                这样用更新的MCTS在来下棋,下好的棋谱再来训练ResNet,如此循环,
                棋力日强.不仅发现了已有的人类定式,甚至发现了新的定式.

                其实看到这里,就会发现其实alphago zero的算法其实没有什么突破,就是用
                resnet替代了之前的SL,rollout以及value network罢了.
                
                但是,最让人震惊
                的在于,即便用机器的对战棋谱,居然还能收敛!!!如果没有强大GPU来验证,
                \textbf{恐怕
                真是贫穷限制了我的想象力.}



            \end{enumerate}

        \end{document}